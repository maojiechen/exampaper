\documentclass[marginline,answers]{BHCexam}


\begin{document}
\biaoti{2023年安徽初中学业水平考试}
\fubiaoti{数学试卷}
\maketitle
%\notice

\begin{questions}


\tiankong
\question $a$是一个四位数,“四舍五入”取近似值为4.68,那么$a$的最大值是\stk{4.6849},最小值是\stk{4.6750}.

\question 等底等高的圆柱和圆锥的体积之差是40立方分米,圆柱的体积是\stk{60}立方分米.

\question 比45千克少$\dfrac{2}{5}$的是\stk{27}千克,\stk{200}千克比150千克多$\dfrac{1}{3}$.

\question 2075立方厘米=\stk{2.075}立方分米,1500平方米=\stk{0.15}公顷.

\question 甲乙两个齿轮齿数比为3:5,它们互相咬合,当甲齿轮转50圈时,乙齿轮转\stk{30}圈.

\question 在一个长8 cm,宽6 cm的长方形里画一个最大的半圆,这个半圆的周长是\stk{20.56}cm,面积是\stk{25.12} cm$^{2}$.

\question 某产品,不合格与合格的个数比是4:6,产品的合格率是\stk{60\%}.

\question 如图是一个等腰直角三角形,它的面积是\stk{4.5}cm$^{2}$,把它以$AB$所在直线为轴旋转一周,形成的图形的体积是\stk{28.26}cm$^{2}$.

\question 0.4:1.6的比值是\stk{0.25},如果前项加上0.8,要使比值不变,后项应加上\stk{3.2}.

\question \stk{扇形}统计图能反映各个部分在总体中所占的百分比.在一个这样的统计图中,某部分占总体的30\%,则该部分扇形的圆心角是\stk{108}$^\circ$.

\question 一个分数的分子增加\stk{20\%},而分母减少\stk{20\%},得到新的分数比原来的分数增加\stk{50\%}.

\question 一件100元的商品,降价5\%后又提价5\%,这时价格为\stk{99.75}.

\question 把一个棱长为8里米的正方形削成一个最大的圆柱体,这个圆柱体的表面积是\stk{301.44}平方厘米,削去的体积是\stk{110.08}立方厘米.

\question 如图,摆一个正六边形需要六根小棒,摆两个正六边形需要11根小棒,按这样摆下去,摆10个正六边形需要\stk{51}根小棒,摆$n$个正六边形需要\stk{5n+1}根小棒.


\xuanze

\question 乐器商店新进了9把小提琴,共花了3600元,售价合理的是\xx{B}

\onech{400把/元}{498元/把}{498把/元}{400元/把}

\question 一个圆和正方形的周长都是12.56厘米,比较它们的面积\xx{C}

\onech{一样大}{正方形大}{圆大}{无法比较}

\question 如图,下列比例式正确的是\xx{B}

\onech{$a:b=c:h$}{$a:h=c:b$}{$b:c=h:a$}{$b:a=c:h$}

\question 如右图,$AE:EB=1:4$,那么甲和乙的面积比是\xx{C}

\onech{2:3}{1:4}{3:2}{4:5}

\question 下列说法中,错误的是\xx{C}

\fourch{某商品打七五折销售,就是比原价降低25\%}{学校在小明家北偏东30$^\circ$方向500米处,小明家在学校西偏南60$^\circ$方向500米处}{当圆柱的底面直径和高相等时,这个圆柱的侧面展开图是一个正方形}{一件衣服150元,先提价10\%,再降价10\%,最后便宜了}



\jiandaa
\question 已知复数~$z$ 满足:$\abs{z}-z^*=\dfrac{10}{1-w\textbf{i}}$(其中~$z^*$
是~$z$ 的共轭复数).
\begin{parts}
\part[7] 求复数~$z$;
\part[7] 若复数~$w=\cos\theta+\textbf{i}\sin\theta\,(\theta\in\mathbb{R})$, 求~$\abs{z-2}$ 的取值范围.
\end{parts}

\begin{solution}
\begin{parts}
\part $z=3+4\textbf{i}$
\part $\abs{z-w}\in[4,6]$
\end{parts}
\end{solution}

\question[14] 函数~$f(x)=4\sin\dfrac{\pi}{12}x\cdot\sin
    \left(\dfrac{\pi}{2}+\dfrac{\pi}{12}x\right),x\in[a,a+1]$,
    其中常数~$a\in[0,5]$, 求函数~$f(x)$ 的最大值~$g(a)$.

\begin{solution}
略
\end{solution}


\jiandab
\question[16] 函数~$f(x)=4\sin\dfrac{\pi}{12}x\cdot\sin
    \left(\dfrac{\pi}{2}+\dfrac{\pi}{12}x\right),x\in[a,a+1]$,
    其中常数~$a\in[0,5]$, 求函数~$f(x)$ 的最大值~$g(a)$.

\begin{solution}
略
\end{solution}


\question 已知复数~$z$ 满足:$\abs{z}-z^*=\dfrac{10}{1-w\textbf{i}}$(其中~$z^*$
是~$z$ 的共轭复数).
\begin{parts}
\part[8] 求复数~$z$;若复数~$w=\cos\theta+\textbf{i}\sin\theta\,(\theta\in\mathbb{R})$, 求~$\abs{z-2}$ 的取值范围.若复数~$w=\cos\theta+\textbf{i}\sin\theta\,(\theta\in\mathbb{R})$, 求~$\abs{z-2}$ 的取值范围.
\part[8] 若复数~$w=\cos\theta+\textbf{i}\sin\theta\,(\theta\in\mathbb{R})$, 求~$\abs{z-2}$ 的取值范围.
\end{parts}

\begin{solution}
\begin{parts}
\part $z=3+4\textbf{i}$
\part $\abs{z-w}\in[4,6]$
\end{parts}
\end{solution}


\jiandac
\question[18] 函数~$f(x)=4\sin\dfrac{\pi}{12}x\cdot\sin
    \left(\dfrac{\pi}{2}+\dfrac{\pi}{12}x\right),x\in[a,a+1]$,
    其中常数~$a\in[0,5]$, 求函数~$f(x)$ 的最大值~$g(a)$.

\begin{solution}
略
\end{solution}

\question[18] 函数~$f(x)=4\sin\dfrac{\pi}{12}x\cdot\sin
\left(\dfrac{\pi}{2}+\dfrac{\pi}{12}x\right),x\in[a,a+1]$,
其中常数~$a\in[0,5]$, 求函数~$f(x)$ 的最大值~$g(a)$.

\jiandad
\question 函数~$f(x)=4\sin\dfrac{\pi}{12}x\cdot\sin
\left(\dfrac{\pi}{2}+\dfrac{\pi}{12}x\right),x\in[a,a+1]$,
其中常数~$a\in[0,5]$, 求函数~$f(x)$ 的最大值~$g(a)$.


\end{questions}

\end{document}
